\section{Cstore\_fdw}

\href{https://citusdata.github.io/cstore\_fdw/}{Cstore\_fdw} расширение реализует модель хранения данных на базе \href{https://en.wikipedia.org/wiki/Column-oriented\_DBMS}{семейства столбцов} (column-oriented systems) для PostgreSQL (колоночное хранение данных). Такое хранение данных обеспечивает заметные преимущества для аналитических задач (\href{https://ru.wikipedia.org/wiki/OLAP}{OLAP}, \href{https://en.wikipedia.org/wiki/Data\_warehouse}{data warehouse}), поскольку требуется считывать меньше данных с диска (благодаря формату хранения и компресии). Расширение использует \href{https://cwiki.apache.org/confluence/display/Hive/LanguageManual+ORC#LanguageManualORC-ORCFileFormat}{Optimized Row Columnar (ORC)} формат для размещения данных на диске, который имеет следующие преимущества:

\begin{itemize}
  \item Уменьшение (сжатие) размера данных в памяти и на диске в 2-4 раза. Можно добавить в расширение другой кодек для сжатия (алгоритм Лемпеля-Зива, LZ присутствует в расширении);
  \item Считывание с диска только тех данных, которые требуются. Повышается производительность по I/O диска для других запросов;
  \item Хранение минимального/максимального значений для групп полей (skip index, индекс с пропусками), что помогает пропустить не требуемые данные на диске при выборке;
\end{itemize}


\subsection{Установка и использование}

Для работы \lstinline!cstore_fdw! требуется \href{https://github.com/protobuf-c/protobuf-c}{protobuf-c} для сериализации и десериализации данных. Далее требуется добавить в \lstinline!postgresql.conf! расширение:

\begin{lstlisting}[language=Bash,label=lst:cstore1,caption=Cstore\_fdw]
shared_preload_libraries = 'cstore_fdw'
\end{lstlisting}

И активировать его для базы:

\begin{lstlisting}[language=SQL,label=lst:cstore2,caption=Cstore\_fdw]
# CREATE EXTENSION cstore_fdw;
\end{lstlisting}

Для загрузки данных в cstore таблицы существует два варианта:

\begin{itemize}
  \item Использование команды \lstinline!COPY! для загрузки или добавления данных из файлов или STDIN;
  \item Использование конструкции \lstinline!INSERT INTO cstore_table SELECT ...! для загрузки или добавления данных из другой таблицы;
\end{itemize}

Cstore таблицы не поддерживают \lstinline!INSERT! (кроме выше упомянутого \lstinline!INSERT INTO ... SELECT!), \lstinline!UPDATE! или \lstinline!DELETE! команды.

Для примера загрузим тестовые данные:

\begin{lstlisting}[language=Bash,label=lst:cstore3,caption=Cstore\_fdw]
$ wget http://examples.citusdata.com/customer_reviews_1998.csv.gz
$ wget http://examples.citusdata.com/customer_reviews_1999.csv.gz

$ gzip -d customer_reviews_1998.csv.gz
$ gzip -d customer_reviews_1999.csv.gz
\end{lstlisting}

Далее загрузим эти данные в cstore таблицу (расширение уже активировано для PostgreSQL):

\begin{lstlisting}[language=SQL,label=lst:cstore4,caption=Cstore таблицы]
-- create server object
CREATE SERVER cstore_server FOREIGN DATA WRAPPER cstore_fdw;

-- create foreign table
CREATE FOREIGN TABLE customer_reviews
(
    customer_id TEXT,
    review_date DATE,
    review_rating INTEGER,
    review_votes INTEGER,
    review_helpful_votes INTEGER,
    product_id CHAR(10),
    product_title TEXT,
    product_sales_rank BIGINT,
    product_group TEXT,
    product_category TEXT,
    product_subcategory TEXT,
    similar_product_ids CHAR(10)[]
)
SERVER cstore_server
OPTIONS(compression 'pglz');

COPY customer_reviews FROM '/tmp/customer_reviews_1998.csv' WITH CSV;
COPY customer_reviews FROM '/tmp/customer_reviews_1999.csv' WITH CSV;

ANALYZE customer_reviews;
\end{lstlisting}

После этого можно проверить как работает расширение:

\begin{lstlisting}[language=SQL,label=lst:cstore5,caption=Cstore запросы]
-- Find all reviews a particular customer made on the Dune series in 1998.
# SELECT
    customer_id, review_date, review_rating, product_id, product_title
FROM
    customer_reviews
WHERE
    customer_id ='A27T7HVDXA3K2A' AND
    product_title LIKE '%Dune%' AND
    review_date >= '1998-01-01' AND
    review_date <= '1998-12-31';
  customer_id   | review_date | review_rating | product_id |                 product_title
----------------+-------------+---------------+------------+-----------------------------------------------
 A27T7HVDXA3K2A | 1998-04-10  |             5 | 0399128964 | Dune (Dune Chronicles (Econo-Clad Hardcover))
 A27T7HVDXA3K2A | 1998-04-10  |             5 | 044100590X | Dune
 A27T7HVDXA3K2A | 1998-04-10  |             5 | 0441172717 | Dune (Dune Chronicles, Book 1)
 A27T7HVDXA3K2A | 1998-04-10  |             5 | 0881036366 | Dune (Dune Chronicles (Econo-Clad Hardcover))
 A27T7HVDXA3K2A | 1998-04-10  |             5 | 1559949570 | Dune Audio Collection
(5 rows)

Time: 238.626 ms

-- Do we have a correlation between a book's title's length and its review ratings?
# SELECT
    width_bucket(length(product_title), 1, 50, 5) title_length_bucket,
    round(avg(review_rating), 2) AS review_average,
    count(*)
FROM
   customer_reviews
WHERE
    product_group = 'Book'
GROUP BY
    title_length_bucket
ORDER BY
    title_length_bucket;
 title_length_bucket | review_average | count
---------------------+----------------+--------
                   1 |           4.26 | 139034
                   2 |           4.24 | 411318
                   3 |           4.34 | 245671
                   4 |           4.32 | 167361
                   5 |           4.30 | 118422
                   6 |           4.40 | 116412
(6 rows)

Time: 1285.059 ms
\end{lstlisting}


\subsection{Заключение}

Более подробно об использовании расширения можно ознакомиться через \href{https://citusdata.github.io/cstore_fdw/}{официальную документацию}.
