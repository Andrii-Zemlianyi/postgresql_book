\section{pgSphere}

\href{http://pgsphere.github.io/}{pgSphere} обеспечивает PostgreSQL сферическими типами данных, а также функциями и операторами для работы с ними. Используется для работы с географическими (может использоваться вместо PostGIS) или астрономическими типами данных.

\subsection{Установка и использование}

Для начала инициализируем расширение в базе данных:

\begin{lstlisting}[language=SQL,label=lst:pgsphereinit,caption=Инициализация pgSphere]
# CREATE EXTENSION pg_sphere;
\end{lstlisting}

После этого можем проверить, что расширение функционирует:

\begin{lstlisting}[language=SQL,label=lst:pgsphereex1,caption=Проверка pgSphere]
# SELECT spoly '{ (270d,-10d), (270d,30d), (290d,10d) } ';
                                                          spoly
-------------------------------------------------------------------------------------------------------------------------
 {(4.71238898038469 , -0.174532925199433),(4.71238898038469 , 0.523598775598299),(5.06145483078356 , 0.174532925199433)}
(1 row)
\end{lstlisting}

И работу индексов:

\begin{lstlisting}[language=SQL,label=lst:pgsphereex2,caption=Проверка pgSphere]
# CREATE TABLE test (
#   pos spoint NOT NULL
# );
CREATE TABLE
# CREATE INDEX test_pos_idx ON test USING GIST (pos);
CREATE INDEX
# INSERT INTO test(pos) VALUES ('( 10.1d, -90d)'), ('( 10d 12m 11.3s, -13d 14m)');
INSERT 0 2
# VACUUM ANALYZE test;
VACUUM
# SELECT set_sphere_output('DEG');
 set_sphere_output
-------------------
 SET DEG
(1 row)

# SELECT * FROM test;
                   pos
------------------------------------------
 (10.1d , -90d)
 (10.2031388888889d , -13.2333333333333d)
(2 rows)
# SET enable_seqscan = OFF;
SET
# EXPLAIN SELECT * FROM test WHERE pos = spoint '(10.1d,-90d)';
                                QUERY PLAN
---------------------------------------------------------------------------
 Bitmap Heap Scan on test  (cost=4.16..9.50 rows=2 width=16)
   Recheck Cond: (pos = '(10.1d , -90d)'::spoint)
   ->  Bitmap Index Scan on test_pos_idx  (cost=0.00..4.16 rows=2 width=0)
         Index Cond: (pos = '(10.1d , -90d)'::spoint)
(4 rows)
\end{lstlisting}

\subsection{Заключение}

Более подробно об использовании расширения можно ознакомиться через \href{http://pgsphere.projects.pgfoundry.org/}{официальную документацию}.
