\section{Prefix}

\href{http://pgfoundry.org/projects/prefix}{Prefix} реализует поиск текста по префиксу (\lstinline!prefix @> text!). Prefix используется в приложениях телефонии, где маршрутизация вызовов и расходы зависят от вызывающего/вызываемого префикса телефонного номера оператора.

\subsection{Установка и использование}

Для начала инициализируем расширение в базе данных:

\begin{lstlisting}[language=SQL,label=lst:pgprefixinit,caption=Инициализация prefix]
# CREATE EXTENSION prefix;
\end{lstlisting}

После этого можем проверить, что расширение функционирует:

\begin{lstlisting}[language=SQL,label=lst:pgprefixexample1,caption=Проверка prefix]
# select '123'::prefix_range @> '123456';
 ?column?
----------
 t
(1 row)

# select a, b, a | b as union, a & b as intersect
  from  (select a::prefix_range, b::prefix_range
           from (values('123', '123'),
                       ('123', '124'),
                       ('123', '123[4-5]'),
                       ('123[4-5]', '123[2-7]'),
                       ('123', '[2-3]')) as t(a, b)
        ) as x;

    a     |    b     |  union   | intersect
----------+----------+----------+-----------
 123      | 123      | 123      | 123
 123      | 124      | 12[3-4]  |
 123      | 123[4-5] | 123      | 123[4-5]
 123[4-5] | 123[2-7] | 123[2-7] | 123[4-5]
 123      | [2-3]    | [1-3]    |
(5 rows)
\end{lstlisting}

В примере~\ref{lst:pgprefixexample2} производится поиск мобильного оператора по номеру телефона:

\begin{lstlisting}[language=SQL,label=lst:pgprefixexample2,caption=Проверка prefix]
$ wget https://github.com/dimitri/prefix/raw/master/prefixes.fr.csv
$ psql

# create table prefixes (
       prefix    prefix_range primary key,
       name      text not null,
       shortname text,
       status    char default 'S',

       check( status in ('S', 'R') )
);
CREATE TABLE
# comment on column prefixes.status is 'S:   - R: reserved';
COMMENT
# \copy prefixes from 'prefixes.fr.csv' with delimiter ';' csv quote '"'
COPY 11966
# create index idx_prefix on prefixes using gist(prefix);
CREATE INDEX
# select * from prefixes limit 10;
 prefix |                            name                            | shortname | status
--------+------------------------------------------------------------+-----------+--------
 010001 | COLT TELECOMMUNICATIONS FRANCE                             | COLT      | S
 010002 | EQUANT France                                              | EQFR      | S
 010003 | NUMERICABLE                                                | NURC      | S
 010004 | PROSODIE                                                   | PROS      | S
 010005 | INTERNATIONAL TELECOMMUNICATION NETWORK France (Vivaction) | ITNF      | S
 010006 | SOCIETE FRANCAISE DU RADIOTELEPHONE                        | SFR       | S
 010007 | SOCIETE FRANCAISE DU RADIOTELEPHONE                        | SFR       | S
 010008 | BJT PARTNERS                                               | BJTP      | S
 010009 | LONG PHONE                                                 | LGPH      | S
 010010 | IPNOTIC TELECOM                                            | TLNW      | S
(10 rows)

# select * from prefixes where prefix @> '0146640123';
 prefix |      name      | shortname | status
--------+----------------+-----------+--------
 0146   | FRANCE TELECOM | FRTE      | S
(1 row)

# select * from prefixes where prefix @> '0100091234';
 prefix |    name    | shortname | status
--------+------------+-----------+--------
 010009 | LONG PHONE | LGPH      | S
(1 row)
\end{lstlisting}


\subsection{Заключение}

Более подробно об использовании расширения можно ознакомиться через \href{https://github.com/dimitri/prefix/blob/master/README.md}{официальную документацию}.
