\section{Pg\_cron}

\href{https://github.com/citusdata/pg\_cron}{Pg\_cron}~--- cron-подобный планировщик задач для PostgreSQL 9.5 или выше, который работает как расширение к базе. Он может выполнять несколько задач параллельно, но одновременно может работать не более одного экземпляра задания (если при запуске задачи предыдущий запуск будет еще выполняться, то запуск будет отложен до выполнения текущей задачи).

\subsection{Установка и использование}

После установки расширения требуется добавить его в \lstinline!postgresql.conf! и перезапустить PostgreSQL:

\begin{lstlisting}[language=Bash,label=lst:pgcron1,caption=pg\_cron]
shared_preload_libraries = 'pg_cron'
\end{lstlisting}

Далее требуется активировать расширение для \lstinline!postgres! базы:

\begin{lstlisting}[language=SQL,label=lst:pgcron2,caption=pg\_cron]
# CREATE EXTENSION pg_cron;
\end{lstlisting}

По умолчанию \lstinline!pg_cron! ожидает, что все таблицы с метаданными будут находиться в \lstinline!postgres! базе данных. Данное поведение можно изменить и указать через параметр \lstinline!cron.database_name! в \lstinline!postgresql.conf! другую базу данных, где \lstinline!pg_cron! будет хранить свои данные.

Внутри \lstinline!pg_cron! использует libpq библиотеку, поэтому потребуется разрешить подключения с \lstinline!localhost! без пароля (\lstinline!trust! в \lstinline!pg_hba.conf!) или же создать \href{https://www.postgresql.org/docs/current/static/libpq-pgpass.html}{.pgpass} файл для настройки подключения к базе.

Для создания cron задач используется функция \lstinline!cron.schedule!:

\begin{lstlisting}[language=SQL,label=lst:pgcron3,caption=Cron.schedule]
-- Delete old data on Saturday at 3:30am (GMT)
SELECT cron.schedule('30 3 * * 6', $$DELETE FROM events WHERE event_time < now() - interval '1 week'$$);
 schedule
----------
       42
\end{lstlisting}

Для удаления созданых задач используется \lstinline!cron.unschedule!:

\begin{lstlisting}[language=SQL,label=lst:pgcron4,caption=Cron.unschedule]
 -- Vacuum every day at 10:00am (GMT)
 SELECT cron.schedule('0 10 * * *', 'VACUUM');
  schedule
 ----------
        43

-- Stop scheduling a job
SELECT cron.unschedule(43);
 unschedule
------------
          t
\end{lstlisting}

В целях безопасности cron задачи выполняются в базе данных, в которой \lstinline!cron.schedule! функция была вызвана с правами доступа текущего пользователя.

Поскольку \lstinline!pg_cron! использует libpq библиотеку, это позволят запускать cron задачи на других базах данных (даже на других серверах). С помощью суперпользователя возможно модифицировать \lstinline!cron.job! таблицу и добавить в нее параметры подключения к другой базе через \lstinline!nodename! и \lstinline!nodeport! поля:

\begin{lstlisting}[language=SQL,label=lst:pgcron5,caption=Cron.job]
INSERT INTO cron.job (schedule, command, nodename, nodeport, database, username)
 VALUES ('0 4 * * *', 'VACUUM', 'worker-node-1', 5432, 'postgres', 'marco');
\end{lstlisting}

В таком случае нужно будет создать \href{https://www.postgresql.org/docs/current/static/libpq-pgpass.html}{.pgpass} файл для настройки подключения к базе на другом сервере.
