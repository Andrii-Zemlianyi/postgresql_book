\section{PLV8}

\href{https://github.com/plv8/plv8}{PLV8} является расширением, которое предоставляет PostgreSQL процедурный язык с движком V8 JavaScript. С помощью этого расширения можно писать в PostgreSQL JavaScript функции, которые можно вызывать из SQL.

\subsection{Установка и использование}

Для начала инициализируем расширение в базе данных:

\begin{lstlisting}[language=SQL,label=lst:plv8jsinit,caption=Инициализация plv8]
# CREATE extension plv8;
\end{lstlisting}

\href{http://en.wikipedia.org/wiki/V8\_(JavaScript\_engine)}{V8} компилирует JavaScript код непосредственно в машинный код и с помощью этого достигается высокая скорость работы. Для примера рассмотрим расчет числа Фибоначчи. Вот функция написана на plpgsql:

\begin{lstlisting}[label=lst:plv8js1,caption=Фибоначчи на plpgsql]
CREATE OR REPLACE FUNCTION
psqlfib(n int) RETURNS int AS $$
 BEGIN
     IF n < 2 THEN
         RETURN n;
     END IF;
     RETURN psqlfib(n-1) + psqlfib(n-2);
 END;
$$ LANGUAGE plpgsql IMMUTABLE STRICT;
\end{lstlisting}

Замерим скорость её работы:

\begin{lstlisting}[label=lst:plv8js2,caption=Скорость расчета числа Фибоначчи на plpgsql]
# SELECT n, psqlfib(n) FROM generate_series(0,25,5) as n;
 n  | psqlfib
----+---------
  0 |       0
  5 |       5
 10 |      55
 15 |     610
 20 |    6765
 25 |   75025
(6 rows)

Time: 2520.386 ms
\end{lstlisting}

Теперь сделаем то же самое, но с использованием PLV8:

\begin{lstlisting}[label=lst:plv8js3,caption=Фибоначчи на plv8]
CREATE OR REPLACE FUNCTION
fib(n int) RETURNS int as $$

  function fib(n) {
    return n<2 ? n : fib(n-1) + fib(n-2)
  }
  return fib(n)

$$ LANGUAGE plv8 IMMUTABLE STRICT;
\end{lstlisting}

Замерим скорость работы:

\begin{lstlisting}[label=lst:plv8js4,caption=Скорость расчета числа Фибоначчи на plv8]
# SELECT n, fib(n) FROM generate_series(0,25,5) as n;
 n  |  fib
----+-------
  0 |     0
  5 |     5
 10 |    55
 15 |   610
 20 |  6765
 25 | 75025
(6 rows)

Time: 5.200 ms
\end{lstlisting}

Как видим, PLV8 приблизительно в 484 (2520.386/5.200) раз быстрее plpgsql. Можно ускорить работу расчета чисел Фибоначи на PLV8 за счет кеширования:

\begin{lstlisting}[label=lst:plv8js5,caption=Фибоначчи на plv8]
CREATE OR REPLACE FUNCTION
fibcache(n int) RETURNS int as $$
  var memo = {0: 0, 1: 1};
  function fib(n) {
    if(!(n in memo))
      memo[n] = fib(n-1) + fib(n-2)
    return memo[n]
  }
  return fib(n);
$$ LANGUAGE plv8 IMMUTABLE STRICT;
\end{lstlisting}

Замерим скорость работы:

\begin{lstlisting}[label=lst:plv8js6,caption=Скорость расчета числа Фибоначчи на plv8]
# SELECT n, fibcache(n) FROM generate_series(0,25,5) as n;
 n  | fibcache
----+----------
  0 |        0
  5 |        5
 10 |       55
 15 |      610
 20 |     6765
 25 |    75025
(6 rows)

Time: 1.202 ms
\end{lstlisting}

Естественно эти измерения не имеют ничего общего с реальным миром (не нужно каждый день считать числа Фибоначчи в базе данных), но позволяет понять, как V8 может помочь ускорить функции, которые занимаются вычислением чего-либо в базе.

\subsection{NoSQL}

Одним из полезных применений PLV8 может быть создание на базе PostgreSQL документоориентированного хранилища. Для хранения неструктурированных данных можно использовать hstore (<<\ref{sec:hstore-extension}~\nameref{sec:hstore-extension}>>), но у него есть свои недостатки:

\begin{itemize}
  \item нет вложенности;
  \item все данные (ключ и значение по ключу) это строка;
\end{itemize}

Для хранения данных многие документно-ориентированные базы данных используют JSON (MongoDB, CouchDB, Couchbase и~т.~д.). Для этого, начиная с PostgreSQL 9.2, добавлен тип данных JSON, а с версии 9.4~--- JSONB. JSON тип можно добавить для PostgreSQL 9.1 и ниже используя PLV8 и \lstinline!DOMAIN!:

\begin{lstlisting}[label=lst:plv8js7,caption=Создание типа JSON]
CREATE OR REPLACE FUNCTION
valid_json(json text)
RETURNS BOOLEAN AS $$
  try {
    JSON.parse(json); return true;
  } catch(e) {
    return false;
  }
$$ LANGUAGE plv8 IMMUTABLE STRICT;

CREATE DOMAIN json AS TEXT
CHECK(valid_json(VALUE));
\end{lstlisting}

Функция \lstinline!valid_json! используется для проверки JSON данных. Пример использования:

\begin{lstlisting}[label=lst:plv8js8,caption=Проверка JSON]
$ CREATE TABLE members ( id SERIAL, profile json );
$ INSERT INTO members
VALUES('not good json');
ERROR:  value for domain json
violates check constraint "json_check"
$ INSERT INTO members
VALUES('{"good": "json", "is": true}');
INSERT 0 1
$ SELECT * FROM members;
	    profile
------------------------------
  {"good": "json", "is": true}
(1 row)
\end{lstlisting}

Рассмотрим пример использования JSON для хранения данных и PLV8 для их поиска. Для начала создадим таблицу и заполним её данными:

\begin{lstlisting}[label=lst:plv8js9,caption=Таблица с JSON полем]
$ CREATE TABLE members ( id SERIAL, profile json );
$ SELECT count(*) FROM members;
  count
---------
 1000000
(1 row)

Time: 201.109 ms
\end{lstlisting}

В \lstinline!profile! поле мы записали приблизительно такую структуру JSON:

\begin{lstlisting}[label=lst:plv8js10,caption=JSON структура]
{                                  +
  "name": "Litzy Satterfield",     +
  "age": 24,                       +
  "siblings": 2,                   +
  "faculty": false,                +
  "numbers": [                     +
    {                              +
      "type":   "work",            +
      "number": "684.573.3783 x368"+
    },                             +
    {                              +
      "type":   "home",            +
      "number": "625.112.6081"     +
    }                              +
  ]                                +
}
\end{lstlisting}

Теперь создадим функцию для вывода значения по ключу из JSON (в данном случае ожидаем цифру):

\begin{lstlisting}[label=lst:plv8js11,caption=Функция для JSON]
CREATE OR REPLACE FUNCTION get_numeric(json_raw json, key text)
RETURNS numeric AS $$
  var o = JSON.parse(json_raw);
  return o[key];
$$ LANGUAGE plv8 IMMUTABLE STRICT;
\end{lstlisting}

Теперь мы можем произвести поиск по таблице, фильтруя по значениям ключей \lstinline!age!, \lstinline!siblings! или другим числовым полям:

\begin{lstlisting}[label=lst:plv8js12,caption=Поиск по данным JSON]
$ SELECT * FROM members WHERE get_numeric(profile, 'age') = 36;
Time: 9340.142 ms
$ SELECT * FROM members WHERE get_numeric(profile, 'siblings') = 1;
Time: 14320.032 ms
\end{lstlisting}

Поиск работает, но скорость очень маленькая. Чтобы увеличить скорость, нужно создать функциональные индексы:

\begin{lstlisting}[label=lst:plv8js13,caption=Создание индексов]
$ CREATE INDEX member_age ON members (get_numeric(profile, 'age'));
$ CREATE INDEX member_siblings ON members (get_numeric(profile, 'siblings'));
\end{lstlisting}

С индексами скорость поиска по JSON станет достаточно высокая:

\begin{lstlisting}[label=lst:plv8js14,caption=Поиск по данным JSON с индексами]
$ SELECT * FROM members WHERE get_numeric(profile, 'age') = 36;
Time: 57.429 ms
$ SELECT * FROM members WHERE get_numeric(profile, 'siblings') = 1;
Time: 65.136 ms
$ SELECT count(*) from members where  get_numeric(profile, 'age') = 26 and get_numeric(profile, 'siblings') = 1;
Time: 106.492 ms
\end{lstlisting}

Получилось отличное документоориентированное хранилище из PostgreSQL. Если используется PostgreSQL 9.4 или новее, то можно использовать JSONB тип, у которого уже есть возможность создавать индексы на требуемые ключи в JSON структуре.

PLV8 позволяет использовать некоторые JavaScript библиотеки внутри PostgreSQL. Вот пример рендера \href{http://mustache.github.com/}{Mustache} темплейтов:

\begin{lstlisting}[label=lst:plv8js15,caption=Функция для рендера Mustache темплейтов]
CREATE OR REPLACE FUNCTION mustache(template text, view json)
RETURNS text as $$
  // …400 lines of mustache.js…
  return Mustache.render(template, JSON.parse(view))
$$ LANGUAGE plv8 IMMUTABLE STRICT;
\end{lstlisting}

\begin{lstlisting}[label=lst:plv8js16,caption=Рендер темплейтов]
$ SELECT mustache(
  'hello {{#things}}{{.}} {{/things}}:) {{#data}}{{key}}{{/data}}',
  '{"things": ["world", "from", "postgresql"], "data": {"key": "and me"}}'
);
		mustache
---------------------------------------
  hello world from postgresql :) and me
(1 row)

Time: 0.837 ms
\end{lstlisting}

Этот пример показывает какие возможности предоставляет PLV8 в PostgreSQL. В действительности рендерить Mustache темплейты в PostgreSQL не лучшая идея.

\subsection{Заключение}

PLV8 расширение предоставляет PostgreSQL процедурный язык с движком V8 JavaScript, с помощью которого можно работать с JavaScript библиотеками, индексировать JSON данные и использовать его как более быстрый язык для вычислений внутри базы.
