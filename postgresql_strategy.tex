\chapter{Стратегии масштабирования для PostgreSQL}

\begin{epigraphs}
\qitem{В конце концов, все решают люди, не стратегии}{Ларри Боссиди}
\end{epigraphs}

\section{Введение}

Многие разработчики крупных проектов сталкиваются с проблемой, когда один-единственный сервер базы данных никак не может справиться с нагрузками. Очень часто такие проблемы происходят из-за неверного проектирования приложения (плохая структура БД для приложения, отсутствие кеширования). Но в данном случае пусть у нас есть <<идеальное>> приложение, для которого оптимизированы все SQL запросы, используется кеширование, PostgreSQL настроен, но все равно не справляется с нагрузкой. Такая проблема может возникнуть как на этапе проектирования, так и на этапе роста приложения. И тут возникает вопрос: какую стратегию выбрать при возникновении подобной ситуации?

Если Ваш заказчик готов купить супер сервер за несколько тысяч долларов (а по мере роста~--- десятков тысяч и т. д.), чтобы сэкономить время разработчиков, но сделать все быстро, можете дальше эту главу не читать. Но такой заказчик~--- мифическое существо и, в основном, такая проблема ложится на плечи разработчиков.

\subsection{Суть проблемы}

Для того, чтобы сделать какой-то выбор, необходимо знать суть проблемы. Существуют два предела, в которые могут уткнуться сервера баз данных:

\begin{itemize}
  \item Ограничение пропускной способности чтения данных;
  \item Ограничение пропускной способности записи данных;
\end{itemize}

Практически никогда не возникает одновременно две проблемы, по крайне мере, это маловероятно (если вы, конечно, не Twitter или Facebook пишете). Если вдруг такое происходит~--- возможно, система неверно спроектирована, и её реализацию следует пересмотреть.

\section{Проблема чтения данных}

Проблема с чтением данных обычно начинается, когда СУБД не в состоянии обеспечить то количество выборок, которое требуется. В основном такое происходит в блогах, новостных лентах и т. д. Хочу сразу отметить, что подобную проблему лучше решать внедрением кеширования, а потом уже думать как масштабировать СУБД.

\subsection{Методы решения}

\begin{itemize}
  \item \textbf{PgPool-II v.3 + PostgreSQL v.9 с Streaming Replication}~--- отличное решение для масштабирования на чтение, более подробно можно ознакомиться по \href{http://pgpool.projects.pgfoundry.org/contrib\_docs/simple\_sr\_setting/index.html}{ссылке}. Основные преимущества:

  \begin{itemize}
    \item Низкая задержка репликации между мастером и слейвом;
    \item Производительность записи падает незначительно;
    \item Отказоустойчивость (failover);
    \item Пулы соединений;
    \item Интеллектуальная балансировка нагрузки~-- проверка задержки репликации между мастером и слейвом (сам проверяет \lstinline!pg_current_xlog_location! и \lstinline!pg_last_xlog_receive_location!);
    \item Добавление слейвов СУБД без остановки pgpool-II;
    \item Простота в настройке и обслуживании;
  \end{itemize}

  \item \textbf{PgPool-II v.3 + PostgreSQL с Slony/Londiste/Bucardo}~--- аналогично предыдущему решению, но с использованием Slony/Londiste/Bucardo. Основные преимущества:

  \begin{itemize}
    \item Отказоустойчивость (failover);
    \item Пулы соединений;
    \item Интеллектуальная балансировка нагрузки~-- проверка задержки репликации между мастером и слейвом;
    \item Добавление слейв СУБД без остановки pgpool-II;
    \item Можно использовать Postgresql ниже 9 версии;
  \end{itemize}

  \item \textbf{Citus}~--- подробнее можно прочитать в <<\ref{sec:citus}~\nameref{sec:citus}>> главе;
  \item \textbf{Postgres-X2}~--- подробнее можно прочитать в <<\ref{sec:postgres-x2}~\nameref{sec:postgres-x2}>> главе;
  \item \textbf{Postgres-XL}~--- подробнее можно прочитать в <<\ref{sec:postgres-xl}~\nameref{sec:postgres-xl}>> главе;
\end{itemize}

\section{Проблема записи данных}

Обычно такая проблема возникает в системах, которые производят анализ больших объемов данных (например аналог Google Analytics). Данные активно пишутся и мало читаются (или читается только суммарный вариант собранных данных).

\subsection{Методы решения}

Один из самых популярных методов решения проблемы~--- размазать нагрузку по времени с помощью систем очередей.

\begin{itemize}
  \item \textbf{PgQ}~--- это система очередей, разработанная на базе PostgreSQL. Разработчики~--- компания Skype. Используется в Londiste (подробнее <<\ref{sec:londiste}~\nameref{sec:londiste}>>). Особенности:

  \begin{itemize}
    \item Высокая производительность благодаря особенностям PostgreSQL;
    \item Общая очередь, с поддержкой нескольких обработчиков и нескольких генераторов событий;
    \item PgQ гарантирует, что каждый обработчик увидит каждое событие как минимум один раз;
    \item События достаются из очереди <<пачками>> (batches);
    \item Чистое API на SQL функциях;
    \item Удобный мониторинг;
  \end{itemize}

  \item \textbf{Citus}~--- подробнее можно прочитать в <<\ref{sec:citus}~\nameref{sec:citus}>> главе;
  \item \textbf{Postgres-X2}~--- подробнее можно прочитать в <<\ref{sec:postgres-x2}~\nameref{sec:postgres-x2}>> главе;
  \item \textbf{Postgres-XL}~--- подробнее можно прочитать в <<\ref{sec:postgres-xl}~\nameref{sec:postgres-xl}>> главе;
\end{itemize}


\section{Заключение}

В данной главе показаны только несколько возможных вариантов решения задач масштабирования PostgreSQL. Таких стратегий существует огромное количество и каждая из них имеет как сильные, так и слабые стороны. Самое важное то, что выбор оптимальной стратегии масштабирования для решения поставленных задач остается на плечах разработчиков и/или администраторов СУБД.
