\section{Проблема чтения данных}

Проблема с чтением данных обычно начинается, когда СУБД не в состоянии обеспечить то количество выборок, которое требуется. В основном такое происходит в блогах, новостных лентах и т.д. Хочу сразу отметить, что подобную проблему лучше решать внедрением кеширования, а потом уже думать как масштабировать СУБД.

\subsection{Методы решения}

\begin{itemize}
  \item \textbf{PgPool-II v.3 + PostgreSQL v.9 с Streaming Replication}~--- отличное решение для масштабирования на чтение, более подробно можно ознакомиться по \href{http://pgpool.projects.pgfoundry.org/contrib\_docs/simple\_sr\_setting/index.html}{ссылке}. Основные преимущества:

  \begin{itemize}
    \item Низкая задержка репликации между мастером и слейвом;
    \item Производительность записи падает незначительно;
    \item Отказоустойчивость (failover);
    \item Пулы соединений;
    \item Интеллектуальная балансировка нагрузки~-- проверка задержки репликации между мастером и слейвом (сам проверяет \lstinline!pg_current_xlog_location! и \lstinline!pg_last_xlog_receive_location!);
    \item Добавление слейвов СУБД без остановки pgpool-II;
    \item Простота в настройке и обслуживании;
  \end{itemize}

  \item \textbf{PgPool-II v.3 + PostgreSQL с Slony/Londiste/Bucardo}~--- аналогично предыдущему решению, но с использованием Slony/Londiste/Bucardo. Основные преимущества:

  \begin{itemize}
    \item Отказоустойчивость (failover);
    \item Пулы соединений;
    \item Интеллектуальная балансировка нагрузки~-- проверка задержки репликации между мастером и слейвом;
    \item Добавление слейв СУБД без остановки pgpool-II;
    \item Можно использовать Postgresql ниже 9 версии;
  \end{itemize}

  \item \textbf{Citus}~--- подробнее можно прочитать в <<\ref{sec:citus}~\nameref{sec:citus}>> главе;
  \item \textbf{Postgres-X2}~--- подробнее можно прочитать в <<\ref{sec:postgres-x2}~\nameref{sec:postgres-x2}>> главе;
  \item \textbf{Postgres-XL}~--- подробнее можно прочитать в <<\ref{sec:postgres-xl}~\nameref{sec:postgres-xl}>> главе;
\end{itemize}
