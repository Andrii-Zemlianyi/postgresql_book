\subsection{Barman}

\href{http://www.pgbarman.org/}{Barman}, как и WAL-E, позволяет создать систему для бэкапа и восстановления PostgreSQL на основе непрерывного резервного копирования. Barman использует для хранения бэкапов отдельный сервер, который может собирать бэкапы как с одного, так и с нескольких PostgreSQL баз данных.

\subsubsection{Установка и настройка}

Рассмотрим простой случай с одним экземпляром PostgreSQL (один сервер) и пусть его хост будет \lstinline!pghost!. Наша задача~--- автоматизировать сбор и хранение бэкапов этой базы на другом сервере (его хост будет \lstinline!brhost!). Для взаимодействия эти два сервера должны быть полностью открыты по SSH (доступ без пароля, по ключам). Для этого можно использовать \lstinline!authorized_keys! файл.

\begin{lstlisting}[language=Bash,label=lst:barman1,caption=Проверка подключения по SSH]
# Проверка подключения с сервера PostgreSQL (pghost)
$ ssh barman@brhost
# Проверка подключения с сервера бэкапов (brhost)
$ ssh postgres@pghost
\end{lstlisting}

Далее нужно установить на сервере для бэкапов barman. Сам barman написан на python и имеет пару зависимостей: python 2.6+, \lstinline!rsync! и python библиотеки (\lstinline!argh!, \lstinline!psycopg2!, \lstinline!python-dateutil!, \lstinline!distribute!). На Ubuntu все зависимости можно поставить одной командой:

\begin{lstlisting}[language=Bash,label=lst:barman2,caption=Установка зависимостей barman]
$ aptitude install python-dev python-argh python-psycopg2 python-dateutil rsync python-setuptools
\end{lstlisting}

Далее нужно установить barman:

\begin{lstlisting}[language=Bash,label=lst:barman3,caption=Установка barman]
$ tar -xzf barman-2.1.tar.gz
$ cd barman-2.1/
$ ./setup.py build
$ sudo ./setup.py install
\end{lstlisting}

Или используя \href{https://wiki.postgresql.org/wiki/Apt}{PostgreSQL Community APT репозиторий}:

\begin{lstlisting}[language=Bash,label=lst:barman-apt1,caption=Установка barman]
$ apt-get install barman
\end{lstlisting}

Теперь перейдем к серверу с PostgreSQL. Для того, чтобы barman мог подключаться к базе данных без проблем, нам нужно выставить настройки доступа в конфигах PostgreSQL:

\begin{lstlisting}[label=lst:barman4,caption=Отредактировать в postgresql.conf]
listen_adress = '*'
\end{lstlisting}

\begin{lstlisting}[label=lst:barman5,caption=Добавить в pg\_hba.conf]
host  all  all  brhost/32  trust
\end{lstlisting}

После этих изменений нужно перегрузить PostgreSQL. Теперь можем проверить с сервера бэкапов подключение к PostgreSQL:

\begin{lstlisting}[language=Bash,label=lst:barman6,caption=Проверка подключения к базе]
$ psql -c 'SELECT version()' -U postgres -h pghost
                                                  version
------------------------------------------------------------------------------------------------------------
 PostgreSQL 9.3.1 on x86_64-unknown-linux-gnu, compiled by gcc (Ubuntu/Linaro 4.7.2-2ubuntu1) 4.7.2, 64-bit
(1 row)
\end{lstlisting}

Далее создадим папку на сервере с бэкапами для хранения этих самых бэкапов:

\begin{lstlisting}[language=Bash,label=lst:barman7,caption=Папка для хранения бэкапов]
$ sudo mkdir -p /srv/barman
$ sudo chown barman:barman /srv/barman
\end{lstlisting}

Для настройки barman создадим \lstinline!/etc/barman.conf!:

\begin{lstlisting}[label=lst:barman8,caption=barman.conf]
[barman]
; Main directory
barman_home = /srv/barman

; Log location
log_file = /var/log/barman/barman.log

; Default compression level: possible values are None (default), bzip2, gzip or custom
compression = gzip

; 'main' PostgreSQL Server configuration
[main]
; Human readable description
description =  "Main PostgreSQL Database"

; SSH options
ssh_command = ssh postgres@pghost

; PostgreSQL connection string
conninfo = host=pghost user=postgres
\end{lstlisting}

Секция <<main>> (так мы назвали для barman наш PostgreSQL сервер) содержит настройки для подключения к PostgreSQL серверу и базе. Проверим настройки:

\begin{lstlisting}[language=Bash,label=lst:barman9,caption=Проверка barman настроек]
$ barman show-server main
Server main:
	active: true
	description: Main PostgreSQL Database
	ssh_command: ssh postgres@pghost
	conninfo: host=pghost user=postgres
	backup_directory: /srv/barman/main
	basebackups_directory: /srv/barman/main/base
	wals_directory: /srv/barman/main/wals
	incoming_wals_directory: /srv/barman/main/incoming
	lock_file: /srv/barman/main/main.lock
	compression: gzip
	custom_compression_filter: None
	custom_decompression_filter: None
	retention_policy: None
	wal_retention_policy: None
	pre_backup_script: None
	post_backup_script: None
	current_xlog: None
	last_shipped_wal: None
	archive_command: None
	server_txt_version: 9.3.1
	data_directory: /var/lib/postgresql/9.3/main
	archive_mode: off
	config_file: /etc/postgresql/9.3/main/postgresql.conf
	hba_file: /etc/postgresql/9.3/main/pg_hba.conf
	ident_file: /etc/postgresql/9.3/main/pg_ident.conf

# barman check main
Server main:
	ssh: OK
	PostgreSQL: OK
	archive_mode: FAILED (please set it to 'on')
	archive_command: FAILED (please set it accordingly to documentation)
	directories: OK
	compression settings: OK
\end{lstlisting}

Все хорошо, вот только PostgreSQL не настроен. Для этого на сервере с PostgreSQL отредактируем конфиг базы:

\begin{lstlisting}[label=lst:barman10,caption=Настройка PostgreSQL]
wal_level = hot_standby # archive для PostgreSQL < 9.0
archive_mode = on
archive_command = 'rsync -a %p barman@brhost:INCOMING_WALS_DIRECTORY/%f'
\end{lstlisting}

где \lstinline!INCOMING_WALS_DIRECTORY!~--- директория для складывания WAL-логов. Её можно узнать из вывода команды \lstinline!barman show-server main!(листинг~\ref{lst:barman9}, указано \lstinline!/srv/barman/main/incoming!). После изменения настроек нужно перегрузить PostgreSQL. Теперь проверим статус на сервере бэкапов:

\begin{lstlisting}[language=Bash,label=lst:barman11,caption=Проверка]
$ barman check main
Server main:
	ssh: OK
	PostgreSQL: OK
	archive_mode: OK
	archive_command: OK
	directories: OK
	compression settings: OK
\end{lstlisting}

Все готово. Для добавления нового сервера процедуру потребуется повторить, а в \lstinline!barman.conf! добавить новый сервер.


\subsubsection{Создание бэкапов}

Получение списка серверов:

\begin{lstlisting}[language=Bash,label=lst:barman12,caption=Список серверов]
$ barman list-server
main - Main PostgreSQL Database
\end{lstlisting}

Запуск создания резервной копии PostgreSQL (сервер указывается последним параметром):

\begin{lstlisting}[language=Bash,label=lst:barman13,caption=Создание бэкапа]
$ barman backup main
Starting backup for server main in /srv/barman/main/base/20121109T090806
Backup start at xlog location: 0/3000020 (000000010000000000000003, 00000020)
Copying files.
Copy done.
Asking PostgreSQL server to finalize the backup.
Backup end at xlog location: 0/30000D8 (000000010000000000000003, 000000D8)
Backup completed
\end{lstlisting}

Такую задачу лучше выполнять раз в сутки (добавить в cron). Посмотреть список бэкапов для указаной базы:

\begin{lstlisting}[language=Bash,label=lst:barman14,caption=Список бэкапов]
$ barman list-backup main
main 20121110T091608 - Fri Nov 10 09:20:58 2012 - Size: 1.0 GiB - WAL Size: 446.0 KiB
main 20121109T090806 - Fri Nov  9 09:08:10 2012 - Size: 23.0 MiB - WAL Size: 477.0 MiB
\end{lstlisting}

Более подробная информация о выбраной резервной копии:

\begin{lstlisting}[language=Bash,label=lst:barman15,caption=Информация о выбраной резервной копии]
$ barman show-backup main 20121110T091608
Backup 20121109T091608:
  Server Name       : main
  Status:           : DONE
  PostgreSQL Version: 90201
  PGDATA directory  : /var/lib/postgresql/9.3/main

  Base backup information:
    Disk usage      : 1.0 GiB
    Timeline        : 1
    Begin WAL       : 00000001000000000000008C
    End WAL         : 000000010000000000000092
    WAL number      : 7
    Begin time      : 2012-11-10 09:16:08.856884
    End time        : 2012-11-10 09:20:58.478531
    Begin Offset    : 32
    End Offset      : 3576096
    Begin XLOG      : 0/8C000020
    End XLOG        : 0/92369120

  WAL information:
    No of files     : 1
    Disk usage      : 446.0 KiB
    Last available  : 000000010000000000000093

  Catalog information:
    Previous Backup : 20121109T090806
    Next Backup     : - (this is the latest base backup)
\end{lstlisting}

Также можно сжимать WAL-логи, которые накапливаются в каталогах командой <<cron>>:

\begin{lstlisting}[language=Bash,label=lst:barman16,caption=Архивирование WAL-логов]
$ barman cron
Processing xlog segments for main
	000000010000000000000001
	000000010000000000000002
	000000010000000000000003
	000000010000000000000003.00000020.backup
	000000010000000000000004
	000000010000000000000005
	000000010000000000000006
\end{lstlisting}

Эту команду требуется добавлять в \lstinline!crontab!. Частота выполнения данной команды зависит от того, как много WAL-логов накапливается (чем больше файлов - тем дольше она выполняется). Barman может сжимать WAL-логи через gzip, bzip2 или другой компрессор данных (команды для сжатия и распаковки задаются через \lstinline!custom_compression_filter! и \lstinline!custom_decompression_filter! соответственно). Также можно активировать компрессию данных при передачи по сети через опцию \lstinline!network_compression! (по умолчанию отключена). Через опции \lstinline!bandwidth_limit! (по умолчанию 0, ограничений нет) и \lstinline!tablespace_bandwidth_limit! возможно ограничить использования сетевого канала.

Для восстановления базы из бэкапа используется команда \lstinline!recover!:

\begin{lstlisting}[language=Bash,label=lst:barman17,caption=Восстановление базы]
$ barman recover --remote-ssh-command "ssh postgres@pghost" main 20121109T090806 /var/lib/postgresql/9.3/main
Starting remote restore for server main using backup 20121109T090806
Destination directory: /var/lib/postgresql/9.3/main
Copying the base backup.
Copying required wal segments.
The archive_command was set to 'false' to prevent data losses.

Your PostgreSQL server has been successfully prepared for recovery!

Please review network and archive related settings in the PostgreSQL
configuration file before starting the just recovered instance.

WARNING: Before starting up the recovered PostgreSQL server,
please review also the settings of the following configuration
options as they might interfere with your current recovery attempt:

    data_directory = '/var/lib/postgresql/9.3/main'		# use data in another directory
    external_pid_file = '/var/run/postgresql/9.3-main.pid'		# write an extra PID file
    hba_file = '/etc/postgresql/9.3/main/pg_hba.conf'	# host-based authentication file
    ident_file = '/etc/postgresql/9.3/main/pg_ident.conf'	# ident configuration file
\end{lstlisting}

Barman может восстановить базу из резервной копии на удаленном сервере через SSH (для этого есть опция \lstinline!remote-ssh-command!). Также barman может восстановить базу, используя \href{http://en.wikipedia.org/wiki/Point-in-time\_recovery}{PITR}: для этого используются опции \lstinline!target-time! (указывается время) или \lstinline!target-xid! (id транзакции).


\subsubsection{Заключение}

Barman помогает автоматизировать сбор и хранение резервных копий PostgreSQL данных на отдельном сервере. Утилита проста, позволяет хранить и удобно управлять бэкапами нескольких PostgreSQL серверов.
