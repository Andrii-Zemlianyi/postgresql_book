\section{Утилиты для тюнинга PostgreSQL}

\subsection{Pgtune}

Для оптимизации настроек для PostgreSQL Gregory Smith создал утилиту \href{http://pgtune.projects.postgresql.org/}{pgtune} в расчёте на обеспечение максимальной производительности для заданной аппаратной конфигурации. Утилита проста в использовании и во многих Linux системах может идти в составе пакетов. Если же нет, можно просто скачать архив и распаковать. Для начала:

\begin{lstlisting}[language=Bash,label=lst:pgtune_settings1,caption=Pgtune]
$ pgtune -i $PGDATA/postgresql.conf -o $PGDATA/postgresql.conf.pgtune
\end{lstlisting}

опцией \lstinline!-i, --input-config! указываем текущий файл postgresql.conf, а \lstinline!-o, --output-config! указываем имя файла для нового postgresql.conf.

Есть также дополнительные опции для настройки конфига:

\begin{itemize}
  \item \lstinline!-M, --memory! используйте этот параметр, чтобы определить общий объем системной памяти. Если не указано, pgtune будет пытаться использовать текущий объем системной памяти;
  \item \lstinline!-T, --type! указывает тип базы данных. Опции: DW, OLTP, Web, Mixed, Desktop;
  \item \lstinline!-c, --connections! указывает максимальное количество соединений. Если он не указан, то будет браться в зависимости от типа базы данных;
\end{itemize}

Существует также \href{http://pgtune.leopard.in.ua/}{онлайн версия pgtune}.

Хочется сразу добавить, что pgtune не <<серебряная пуля>> для оптимизации настройки PostgreSQL. Многие настройки зависят не только от аппаратной конфигурации, но и от размера базы данных, числа соединений и сложности запросов, так что оптимально настроить базу данных возможно, только учитывая все эти параметры.


\subsection{pg\_buffercache}

\href{http://www.postgresql.org/docs/current/static/pgbuffercache.html}{Pg\_buffercache}~--- расширение для PostgreSQL, которое позволяет получить представление об использовании общего буфера (\lstinline!shared_buffer!) в базе. Расширение позволяет взглянуть какие из данных кэширует база, которые активно используются в запросах. Для начала нужно установить расширение:

\begin{lstlisting}[language=SQL,label=lst:pgbuffercache1,caption=pg\_buffercache]
# CREATE EXTENSION pg_buffercache;
\end{lstlisting}

Теперь доступно \lstinline!pg_buffercache! представление, которое содержит:

\begin{itemize}
  \item \lstinline!bufferid!~--- ID блока в общем буфере;
  \item \lstinline!relfilenode!~--- имя папки, где данные расположены;
  \item \lstinline!reltablespace!~--- Oid таблицы;
  \item \lstinline!reldatabase!~--- Oid базы данных;
  \item \lstinline!relforknumber!~--- номер ответвления;
  \item \lstinline!relblocknumber!~--- номер страницы;
  \item \lstinline!isdirty!~--- грязная страница?;
  \item \lstinline!usagecount!~--- количество LRU страниц;
\end{itemize}

ID блока в общем буфере (\lstinline!bufferid!) соответствует количеству используемого буфера таблицей, индексом, прочим. Общее количество доступных буферов определяется двумя вещами:

\begin{itemize}
  \item Размер буферного блока. Этот размер блока определяется опцией \lstinline!--with-blocksize! при конфигурации. Значение по умолчанию~--- 8 КБ, что достаточно в большинстве случаев, но его возможно увеличить или уменьшить в зависимости от ситуации. Для того чтобы изменить это значение, необходимо будет перекомпилировать PostgreSQL;
  \item Размер общего буфера. Определяется опцией \lstinline!shared_buffers! в PostgreSQL конфиге.
\end{itemize}

Например, при использовании \lstinline!shared_buffers! в 128 МБ с 8 КБ размера блока получится 16384 буферов. Представление pg\_buffercache будет иметь такое же число строк~--- 16384. С \lstinline!shared_buffers! в 256 МБ и размером блока в 1 КБ получим 262144 буферов.

Для примера рассмотрим простой запрос показывающий использование буферов объектами (таблицами, индексами, прочим):

\begin{lstlisting}[language=SQL,label=lst:pgbuffercache2,caption=pg\_buffercache]
# SELECT c.relname, count(*) AS buffers
FROM pg_buffercache b INNER JOIN pg_class c
ON b.relfilenode = pg_relation_filenode(c.oid) AND
b.reldatabase IN (0, (SELECT oid FROM pg_database WHERE datname = current_database()))
GROUP BY c.relname
ORDER BY 2 DESC
LIMIT 10;

             relname             | buffers
---------------------------------+---------
 pgbench_accounts                |    4082
 pgbench_history                 |      53
 pg_attribute                    |      23
 pg_proc                         |      14
 pg_operator                     |      11
 pg_proc_oid_index               |       9
 pg_class                        |       8
 pg_attribute_relid_attnum_index |       7
 pg_proc_proname_args_nsp_index  |       6
 pg_class_oid_index              |       5
(10 rows)
\end{lstlisting}

Этот запрос показывает объекты (таблицы и индексы) в кэше:

\begin{lstlisting}[language=SQL,label=lst:pgbuffercache3,caption=pg\_buffercache]
# SELECT c.relname, count(*) AS buffers,usagecount
 FROM pg_class c
 INNER JOIN pg_buffercache b
 ON b.relfilenode = c.relfilenode
 INNER JOIN pg_database d
 ON (b.reldatabase = d.oid AND d.datname = current_database())
GROUP BY c.relname,usagecount
ORDER BY c.relname,usagecount;

             relname              | buffers | usagecount
----------------------------------+---------+------------
 pg_rewrite                       |       3 |          1
 pg_rewrite_rel_rulename_index    |       1 |          1
 pg_rewrite_rel_rulename_index    |       1 |          2
 pg_statistic                     |       1 |          1
 pg_statistic                     |       1 |          3
 pg_statistic                     |       2 |          5
 pg_statistic_relid_att_inh_index |       1 |          1
 pg_statistic_relid_att_inh_index |       3 |          5
 pgbench_accounts                 |    4082 |          2
 pgbench_accounts_pkey            |       1 |          1
 pgbench_history                  |      53 |          1
 pgbench_tellers                  |       1 |          1
\end{lstlisting}

Это запрос показывает какой процент общего буфера используют обьекты (таблицы и индексы) и на сколько процентов объекты находятся в самом кэше (буфере):

\begin{lstlisting}[language=SQL,label=lst:pgbuffercache4,caption=pg\_buffercache]
# SELECT
 c.relname,
 pg_size_pretty(count(*) * 8192) as buffered,
 round(100.0 * count(*) /
 (SELECT setting FROM pg_settings WHERE name='shared_buffers')::integer,1)
 AS buffers_percent,
 round(100.0 * count(*) * 8192 / pg_table_size(c.oid),1)
 AS percent_of_relation
FROM pg_class c
 INNER JOIN pg_buffercache b
 ON b.relfilenode = c.relfilenode
 INNER JOIN pg_database d
 ON (b.reldatabase = d.oid AND d.datname = current_database())
GROUP BY c.oid,c.relname
ORDER BY 3 DESC
LIMIT 20;

-[ RECORD 1 ]-------+---------------------------------
 relname             | pgbench_accounts
 buffered            | 32 MB
 buffers_percent     | 24.9
 percent_of_relation | 99.9
-[ RECORD 2 ]-------+---------------------------------
 relname             | pgbench_history
 buffered            | 424 kB
 buffers_percent     | 0.3
 percent_of_relation | 94.6
-[ RECORD 3 ]-------+---------------------------------
 relname             | pg_operator
 buffered            | 88 kB
 buffers_percent     | 0.1
 percent_of_relation | 61.1
-[ RECORD 4 ]-------+---------------------------------
 relname             | pg_opclass_oid_index
 buffered            | 16 kB
 buffers_percent     | 0.0
 percent_of_relation | 100.0
-[ RECORD 5 ]-------+---------------------------------
 relname             | pg_statistic_relid_att_inh_index
 buffered            | 32 kB
 buffers_percent     | 0.0
 percent_of_relation | 100.0
\end{lstlisting}

Используя эти данные можно проанализировать для каких объектов не хватает памяти или какие из них потребляют основную часть общего буфера. На основе этого можно более правильно настраивать \lstinline!shared_buffers! параметр для PostgreSQL.
