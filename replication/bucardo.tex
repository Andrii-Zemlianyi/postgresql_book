\section{Bucardo}

\href{https://bucardo.org/wiki/Bucardo}{Bucardo}~--- асинхронная master-master или master-slave репликация PostgreSQL, которая написана на Perl. Система очень гибкая, поддерживает несколько видов синхронизации и обработки конфликтов.

\subsection{Установка}

Установка будет проводиться на Debian сервере. Сначала нужно установить \lstinline!DBIx::Safe Perl! модуль.

\begin{lstlisting}[language=Bash,label=lst:bucardo1,caption=Установка]
$ apt-get install libdbix-safe-perl
\end{lstlisting}

Для других систем можно поставить из \href{http://search.cpan.org/CPAN/authors/id/T/TU/TURNSTEP/}{исходников}:

\begin{lstlisting}[language=Bash,label=lst:bucardo2,caption=Установка]
$ tar xvfz dbix_safe.tar.gz
$ cd DBIx-Safe-1.2.5
$ perl Makefile.PL
$ make
$ make test
$ sudo make install
\end{lstlisting}

Теперь ставим сам Bucardo. \href{http://bucardo.org/wiki/Bucardo#Obtaining_Bucardo}{Скачиваем} его и инсталлируем:

\begin{lstlisting}[language=Bash,label=lst:bucardo3,caption=Установка]
$ wget http://bucardo.org/downloads/Bucardo-5.4.1.tar.gz
$ tar xvfz Bucardo-5.4.1.tar.gz
$ cd Bucardo-5.4.1
$ perl Makefile.PL
$ make
$ sudo make install
\end{lstlisting}

Для работы Bucardo потребуется установить поддержку pl/perl языка в PostgreSQL.

\begin{lstlisting}[language=Bash,label=lst:bucardo4,caption=Установка]
$ sudo aptitude install postgresql-plperl-9.5
\end{lstlisting}

и дополнительные пакеты для Perl (DBI, DBD::Pg, Test::Simple, boolean):

\begin{lstlisting}[language=Bash,label=lst:bucardo-packet1,caption=Установка]
$ sudo aptitude install libdbd-pg-perl libboolean-perl
\end{lstlisting}

Теперь можем приступать к настройке репликации.

\subsection{Настройка}

\subsubsection{Инициализация Bucardo}

Запускаем установку Bucardo:

\begin{lstlisting}[language=Bash,label=lst:bucardo5,caption=Инициализация Bucardo]
$ bucardo install
\end{lstlisting}

Во время установки будут показаны настройки подключения к PostgreSQL, которые можно будет изменить:

\begin{lstlisting}[language=Bash,label=lst:bucardo6,caption=Инициализация Bucardo]
This will install the bucardo database into an existing Postgres cluster.
Postgres must have been compiled with Perl support,
and you must connect as a superuser

We will create a new superuser named 'bucardo',
and make it the owner of a new database named 'bucardo'

Current connection settings:
1. Host:          <none>
2. Port:          5432
3. User:          postgres
4. Database:      postgres
5. PID directory: /var/run/bucardo
\end{lstlisting}

После подтверждения настроек, Bucardo создаст пользователя \lstinline!bucardo! и базу данных \lstinline!bucardo!.
Данный пользователь должен иметь право логиниться через Unix socket, поэтому лучше заранее дать ему такие права в \lstinline!pg_hda.conf!.

После успешной установки можно проверить конфигурацию через команду \lstinline!bucardo show all!:

\begin{lstlisting}[language=Bash,label=lst:bucardo-status1,caption=Инициализация Bucardo]
$ bucardo show all
autosync_ddl              = newcol
bucardo_initial_version   = 5.0.0
bucardo_vac               = 1
bucardo_version           = 5.0.0
ctl_checkonkids_time      = 10
ctl_createkid_time        = 0.5
ctl_sleep                 = 0.2
default_conflict_strategy = bucardo_latest
default_email_from        = nobody@example.com
default_email_host        = localhost
default_email_to          = nobody@example.com
...
\end{lstlisting}

\subsubsection{Настройка баз данных}

Теперь нужно настроить базы данных, с которыми будет работать Bucardo. Обозначим базы как \lstinline!master_db! и \lstinline!slave_db!. Реплицировать будем \lstinline!simple_database! базу. Сначала настроим мастер базу:

\begin{lstlisting}[language=Bash,label=lst:bucardo7,caption=Настройка баз данных]
$ bucardo add db master_db dbname=simple_database host=master_host
Added database "master_db"
\end{lstlisting}

Данной командой указали базу данных и дали ей имя \lstinline!master_db! (для того, что в реальной жизни \lstinline!master_db! и \lstinline!slave_db! имеют одинаковое название базы \lstinline!simple_database! и их нужно отличать в Bucardo).

Дальше добавляем \lstinline!slave_db!:

\begin{lstlisting}[language=Bash,label=lst:bucardo8,caption=Настройка баз данных]
$ bucardo add db slave_db dbname=simple_database port=5432 host=slave_host
\end{lstlisting}

\subsubsection{Настройка репликации}

Теперь требуется настроить синхронизацию между этими базами данных. Делается это командой \lstinline!sync!:

\begin{lstlisting}[language=Bash,label=lst:bucardo9,caption=Настройка репликации]
$ bucardo add sync delta dbs=master_db:source,slave_db:target conflict_strategy=bucardo_latest tables=all
Added sync "delta"
Created a new relgroup named "delta"
Created a new dbgroup named "delta"
  Added table "public.pgbench_accounts"
  Added table "public.pgbench_branches"
  Added table "public.pgbench_history"
  Added table "public.pgbench_tellers"
\end{lstlisting}

Данная команда устанавливает Bucardo триггеры в PostgreSQL для master-slave репликации. Значения параметров:

\begin{itemize}
  \item \lstinline!dbs!~--- список баз данных, которые следует синхронизировать. Значение \lstinline!source! или \lstinline!target! указывает, что это master или slave база данных соответственно (их может быть больше одной);

  \item \lstinline!conflict_strategy!~--- для работы в режиме master-master нужно указать как Bucardo должен решать конфликты синхронизации. Существуют следующие стратегии:

  \begin{itemize}
    \item \lstinline!bucardo_source!~--- при конфликте мы копируем данные с source;
    \item \lstinline!bucardo_target!~--- при конфликте мы копируем данные с target;
    \item \lstinline!bucardo_skip!~--- конфликт мы просто не реплицируем. Не рекомендуется для продакшен систем;
    \item \lstinline!bucardo_random!~--- каждая БД имеет одинаковый шанс, что её изменение будет взято для решение конфликта;
    \item \lstinline!bucardo_latest!~--- запись, которая была последней изменена решает конфликт;
    \item \lstinline!bucardo_abort!~--- синхронизация прерывается;
  \end{itemize}

  \item \lstinline!tables!~--- таблицы, которые требуется синхронизировать. Через <<all>> указываем все;
\end{itemize}

Для master-master репликации требуется выполнить:

\begin{lstlisting}[language=Bash,label=lst:bucardo10,caption=Настройка репликации]
$ bucardo add sync delta dbs=master_db:source,slave_db:source conflict_strategy=bucardo_latest tables=all
\end{lstlisting}

Пример для создания master-master и master-slave репликации:

\begin{lstlisting}[language=Bash,label=lst:bucardo-master-slave1,caption=Настройка репликации]
$ bucardo add sync delta dbs=master_db1:source,master_db2:source,slave_db1:target,slave_db2:target conflict_strategy=bucardo_latest tables=all
\end{lstlisting}

Для проверки состояния репликации:

\begin{lstlisting}[language=Bash,label=lst:bucardo-master-slave2,caption=Проверка состояния репликации]
$ bucardo status
PID of Bucardo MCP: 12122
 Name    State    Last good    Time     Last I/D    Last bad    Time
=======+========+============+========+===========+===========+=======
 delta | Good   | 13:28:53   | 13m 6s | 3685/7384 | none      |
\end{lstlisting}

\subsubsection{Запуск/Остановка репликации}

Запуск репликации:

\begin{lstlisting}[language=Bash,label=lst:bucardo11,caption=Запуск репликации]
$ bucardo start
\end{lstlisting}

Остановка репликации:

\begin{lstlisting}[language=Bash,label=lst:bucardo12,caption=Остановка репликации]
$ bucardo stop
\end{lstlisting}

\subsection{Общие задачи}

\subsubsection{Просмотр значений конфигурации}

\begin{lstlisting}[language=Bash,label=lst:bucardo13,caption=Просмотр значений конфигурации]
$ bucardo show all
\end{lstlisting}

\subsubsection{Изменения значений конфигурации}

\begin{lstlisting}[language=Bash,label=lst:bucardo14,caption=Изменения значений конфигурации]
$ bucardo set name=value
\end{lstlisting}

Например:

\begin{lstlisting}[language=Bash,label=lst:bucardo15,caption=Изменения значений конфигурации]
$ bucardo_ctl set syslog_facility=LOG_LOCAL3
\end{lstlisting}

\subsubsection{Перегрузка конфигурации}

\begin{lstlisting}[language=Bash,label=lst:bucardo16,caption=Перегрузка конфигурации]
$ bucardo reload_config
\end{lstlisting}

Более подробную информацию можно найти на \href{http://bucardo.org/}{официальном сайте}.


\subsection{Репликация в другие типы баз данных}

Начиная с версии 5.0 Bucardo поддерживает репликацию в другие источники данных: drizzle, mongo, mysql, oracle, redis и sqlite (тип базы задается при использовании команды \lstinline!bucardo add db! через ключ <<type>>, который по умолчанию postgres). Давайте рассмотрим пример с redis базой. Для начала потребуется установить redis адаптер для Perl (для других баз устанавливаются соответствующие):

\begin{lstlisting}[language=Bash,label=lst:bucardo-redis1,caption=Установка redis]
$ aptitude install libredis-perl
\end{lstlisting}

Далее зарегистрируем redis базу в Bucardo:

\begin{lstlisting}[language=Bash,label=lst:bucardo-redis2,caption=Добавление redis базы]
$ bucardo add db R dbname=simple_database type=redis
Added database "R"
\end{lstlisting}

Создадим группу баз данных под названием \lstinline!pg_to_redis!:

\begin{lstlisting}[language=Bash,label=lst:bucardo-redis3,caption=Группа баз данных]
$ bucardo add dbgroup pg_to_redis master_db:source slave_db:source R:target
Created dbgroup "pg_to_redis"
Added database "master_db" to dbgroup "pg_to_redis" as source
Added database "slave_db" to dbgroup "pg_to_redis" as source
Added database "R" to dbgroup "pg_to_redis" as target
\end{lstlisting}

И создадим репликацию:

\begin{lstlisting}[language=Bash,label=lst:bucardo-redis4,caption=Установка sync]
$ bucardo add sync pg_to_redis_sync tables=all dbs=pg_to_redis status=active
Added sync "pg_to_redis_sync"
  Added table "public.pgbench_accounts"
  Added table "public.pgbench_branches"
  Added table "public.pgbench_history"
  Added table "public.pgbench_tellers"
\end{lstlisting}

После перезапуска Bucardo данные с PostgreSQL таблиц начнут реплицироваться в Redis:

\begin{lstlisting}[language=Bash,label=lst:bucardo-redis5,caption=Репликация в redis]
$ pgbench -T 10 -c 5 simple_database
$ redis-cli monitor
"HMSET" "pgbench_history:6" "bid" "2" "aid" "36291" "delta" "3716" "mtime" "2014-07-11 14:59:38.454824" "hid" "4331"
"HMSET" "pgbench_history:2" "bid" "1" "aid" "65179" "delta" "2436" "mtime" "2014-07-11 14:59:38.500896" "hid" "4332"
"HMSET" "pgbench_history:14" "bid" "2" "aid" "153001" "delta" "-264" "mtime" "2014-07-11 14:59:38.472706" "hid" "4333"
"HMSET" "pgbench_history:15" "bid" "1" "aid" "195747" "delta" "-1671" "mtime" "2014-07-11 14:59:38.509839" "hid" "4334"
"HMSET" "pgbench_history:3" "bid" "2" "aid" "147650" "delta" "3237" "mtime" "2014-07-11 14:59:38.489878" "hid" "4335"
"HMSET" "pgbench_history:15" "bid" "1" "aid" "39521" "delta" "-2125" "mtime" "2014-07-11 14:59:38.526317" "hid" "4336"
"HMSET" "pgbench_history:14" "bid" "2" "aid" "60105" "delta" "2555" "mtime" "2014-07-11 14:59:38.616935" "hid" "4337"
"HMSET" "pgbench_history:15" "bid" "2" "aid" "186655" "delta" "930" "mtime" "2014-07-11 14:59:38.541296" "hid" "4338"
"HMSET" "pgbench_history:15" "bid" "1" "aid" "101406" "delta" "668" "mtime" "2014-07-11 14:59:38.560971" "hid" "4339"
"HMSET" "pgbench_history:15" "bid" "2" "aid" "126329" "delta" "-4236" "mtime" "2014-07-11 14:59:38.5907" "hid" "4340"
"DEL" "pgbench_tellers:20"
\end{lstlisting}

Данные в Redis хранятся в виде хешей:

\begin{lstlisting}[language=Bash,label=lst:bucardo-redis6,caption=Данные в redis]
$ redis-cli "HGETALL" "pgbench_history:15"
 1) "bid"
 2) "2"
 3) "aid"
 4) "126329"
 5) "delta"
 6) "-4236"
 7) "mtime"
 8) "2014-07-11 14:59:38.5907"
 9) "hid"
10) "4340"
\end{lstlisting}

Также можно проверить состояние репликации:

\begin{lstlisting}[language=Bash,label=lst:bucardo-redis6,caption=Установка redis]
$ bucardo status
PID of Bucardo MCP: 4655
 Name               State    Last good    Time     Last I/D    Last bad    Time
==================+========+============+========+===========+===========+========
 delta            | Good   | 14:59:39   | 8m 15s | 0/0       | none      |
 pg_to_redis_sync | Good   | 14:59:40   | 8m 14s | 646/2546  | 14:59:39  | 8m 15s
\end{lstlisting}

Теперь данные из redis могут использоваться для приложения в виде быстрого кэш хранилища.
